% Macros de estilo
% Definimos nuevos comandos para facilitar la uniformidad de estilos.
% Ayudará, por supuesto, crear atajos de teclado que los introduzcan, o
% tener un editor que ayude con el autocompletado.

% Palabras en inglés: \ingles
% Definimos tanto el estilo general como el idioma. Lo del idioma es necesario
% porque el paquete de idioma define muchas cuestiones. Ver la documentación
% del paquete Polyglossia.
% Usamos cursivas (_italics_) para los extranjerismos.
\newcommand{\ingles}[1]{\textit{\texten{#1}}}


% Nombres propios: \nombre
% Ya sean personas o instituciones. Nombres no genéricos. Títulos de libros o documentos no.
\newcommand{\nombre}[1]{\textsf{#1}}

\newcommand{\programa}[1]{\texttt{#1}}

\def\td{\todo[inline]}



%\begin{figure}[hbtp]
%\centering
%\includegraphics[width=0.4\linewidth]{dummy}
%\caption[]{\td{Cita}}
%\label{fig:}
%\end{figure}


\def\imagenchica{5cm}
\def\imagenmedia{10cm}
\def\imagengrande{15cm}


\newcommand{\criterio}[3]{%
  \noindent
  \begin{minipage}{0.59\textwidth}
  \begin{description}
  \item[#1] #3
  \end{description}
  \end{minipage}\hfill
  \begin{minipage}{0.4\textwidth}
  #2
  \end{minipage}
  \medskip
  }
  
\newcommand{\criterioo}[3]{%
  \noindent
  \begin{minipage}{0.59\textwidth}
    \noindent
    \begin{description}
      \item[#1] #3
    \end{description}
  \end{minipage}\hfill
  \begin{minipage}{0.4\textwidth}
    \noindent
    \centering
    \includegraphics[]{imgs/#2}
  \end{minipage}
  \bigskip
}
  
\newcommand{\criteriooo}[4]{%
  \noindent
  \begin{minipage}{0.59\textwidth}
    \noindent
    \begin{description}
      \item[#1] #4
    \end{description}
  \end{minipage}\hfill
  \begin{minipage}{0.4\textwidth}
    \noindent
    \centering
    \includegraphics[]{imgs/#2}
    \includegraphics[]{imgs/#3}
  \end{minipage}
  \bigskip
}

\def\escaladiagramas{0.25}

%\includegraphics[width=\textwidth,height=0.9\textheight,keepaspectratio]{SegmentadorBorde1.png}