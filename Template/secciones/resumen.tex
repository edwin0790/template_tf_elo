\begin{abstract}
\todo{Provisorio}

En este trabajo se diseñó y creó un programa informático para la clasificación automática de almendras peladas a partir de imágenes de ellas, analizando diversas características de forma y de color. Para probarlo se construyó un prototipo de sistema de visión artificial con el cual se creó un conjunto de \num{564} imágenes de almendras y otros objetos. El conjunto de imágenes fue etiquetado manualmente en base a las normas de la Comisión Económica de las Naciones Unidas para Europa (UNECE) para el comercio de almendras. Los resultados del clasificador desarrollado son similares a los obtenidos con algoritmos de clasificación estándar, como máquinas de soporte vectorial o \ingles{Boosted Trees}. Los descriptores elegidos permiten clasificar binariamente el conjunto con una exactitud global de \SI{93}{\percent}.

\end{abstract}
