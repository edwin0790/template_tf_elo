%  PLANTILLA TRABAJO FINAL ELO
%
%  Esta es una plantilla de LaTeX para el trabajo final de graduación para la carrera de 
%  Ingeniería Electrónica de la Universidad Nacional de San Juan. 
%  NO ES OFICIAL. Está basada en las normas propuestas (y exigidas) por la Comisión de 
%  Trabajo Final de carrera.
%
%  COLABORADORES
%  2017 
%    Edwin Barragán - ELO 23714
%    Pablo Aguado - ELO 23724
%
%  ---------------------------------------------------------------------------------------
%
%  Intentamos que toda decisión de diseño esté justificada y que toda configuración esté
%  comentada. Además incluimos comentarios útiles para quienes se están iniciando en LaTeX.
%
%  Usamos Koma-Script como clase base. La clase de este documento es Koma-Script Report.
%  A continuación, algunas recomendaciones.
%  
%  FUENTES DE INFORMACIÓN
%    LATEX
%  1. Manual de LaTeX en Wikibooks
%      Español  https://es.wikibooks.org/wiki/Manual_de_LaTeX (está menos completo que el
%                                                                                                      que está inglés, podés ayudar
%                                                                                                                             a la traducción)
%      Inglés https://en.wikibooks.org/wiki/LaTeX
%  2. TEX en StackExchange  https://tex.stackexchange.com/


%  ---------------------------------------------------------------------------------------
%  CONFIGURACIÓN
% Incluimos archivos de configuración.
% \input{archivo} sirve para incluir textualmente un archivo (como un #include en C)

% Lo estrictamente determinado por las Normas Propuestas
% Clase del documento. Koma-Script Report de base.
\documentclass[draft,spanish,twoside=no]{scrreprt}
%\documentclass[final,spanish,twoside=no]{scrreprt}
\KOMAoptions{numbers=noendperiod} %NO ESTA FUNCIONANDO
%\KOMAoptions{headinclude=true}
%\KOMAoptions{footinclude=true}
%\KOMAoptions{DIV=15}



% IDIOMA
% Polyglossia - An alternative to babel for XeLaTeX and LuaLaTeX
% https://ctan.org/pkg/polyglossia
% Este paquete es para mejor soporte de idiomas.
%\usepackage[]{polyglossia}
%\setdefaultlanguage[]{spanish}
%\setotherlanguage[variant=american]{english}
% Para escribir en inglés, usar el comando \textenglish{texto}
% o el entorno english: \begin{english}[opciones]{}\end{english}

\usepackage[utf8]{inputenc}
% Make latex understand and use the typographic
% rules of the language used in the document.
\usepackage[main=spanish, english]{babel}
\babeltags{en = english} % Y entonces \texten{texto} servirá para inglés. igual usaremos macro \ingles{texto}.
%Para cosas largas: \begin{en} y \end{en}

%\addto\captionsspanish{
%	\def\tablename{Tabla}
%	\def\listtablename{\'Indice de tablas}
%}


% Para usar 'tablas' en lugar de 'cuadros'.
% Alternativa si usamos Koma-Script:
\renewcaptionname{spanish}{\listtablename}{Índice de tablas}
\renewcaptionname{spanish}{\tablename}{Tabla}
% Alternativa propuesta por Polyglossia:
%\usepackage{etoolbox}
%\gappto\captionsspanish{\renewcommand{\tablename}{Tabla}}
%\gappto\captionsspanish{\renewcommand{\listtablename}{Índice de tablas}}



%Tamaño de papel: A4 

%Márgenes: Superior = Inferior = 2,5cm; Izquierdo = 2cm; Derecho = 2cm 
\usepackage[left=2cm,top=2.5cm,right=2cm,bottom=2.5cm,bindingoffset=0cm,footskip=1.5cm]{geometry}

%Impresión en doble faz.
% Está arriba con twoside=no

%Letra tamaño 12. Tipo Arial o Times New Roman. 
%\usepackage{helvet}
%\usepackage[scaled]{helvet} 
%Helvetica is actually somewhat larger than other typefaces of the same nominal size.  As a result, mixing, e.g., Times and Helvetica within running text may look bad. This  can  be  fixed  by  loading  the  package  with  the  option [scaled=〈scale〉], for instance: \usepackage[scaled=.92]{helvet}.  As a result, the font family phv (Helvetica) will be scaled down to 92% of its ‘natural’ size, which is suitable for use with Adobe Times.  Specifying [scaled] alone is equivalent to [scaled=0.95].

\usepackage{newtxtext, newtxmath} % JUSTIFICAR
\usepackage[zerostyle=d,scaled=.95]{newtxtt}
%\usepackage[scaled=.8]{beramono}
%\renewcommand{\familydefault}{\sfdefault}
\KOMAoptions{fontsize=12pt}

%Espaciado de renglones: 1 (Espaciado simple). 
%\usepackage[singlespacing]{setspace} 
% KOMA lo tiene por defecto


%Espaciado de párrafos: un renglón libre.Cada punto aparte genera un párrafo. //Entonces no habrá sangrías.
\KOMAoptions{parskip=full}

%Ecuaciones  centradas.  Deben  aparecer  después  de  ser  citadas  en  el  texto. Numeración entre paréntesis justificada a la derecha. 

%Todas  las  figuras  deben  estar  centradas,  tener  un  Nº  de  figura  y  leyenda.  Deben aparecer después de ser citadas en el texto. 

%Encabezado de página: en todas las páginas exceptuando los inicios de capítulo. El encabezado debe incluir el número y título de capítulo en cursiva tamaño 9 y estar justificado a la izquierda.
\usepackage[automark,headsepline,draft=false]{scrlayer-scrpage}
\clearpairofpagestyles
\lohead{\headmark}
\setkomafont{pageheadfoot}{\itshape\footnotesize} % scriptsize es 8 si base es 12. Else: footnotesize, que es 10.
\pagestyle{scrheadings}

%Numeración  de  páginas  en  el  pié  de  página,  justificada  a  la  derecha,  número  no cursivo, tamaño 12. 
\ofoot[\pagemark]{\pagemark} % opcion incluye las planas
%\ofoot[]{\pagemark}
\setkomafont{pagenumber}{\upshape\normalsize} %12

%Tamaño de títulos:  
%//Quedan muy chicos. Deberíamos usar los sugeridos por defecto.
%De capítulo    16 en negrita 
\setkomafont{chapter}{\bfseries\Large} % 17,28pt

%De secciones  14 en negrita 
\setkomafont{section}{\bfseries\large} % 14,4pt

%Secundarios   14 en negrita 
\setkomafont{subsection}{\bfseries\large} % 14,4pt


%No referenciados  12 en negrita
%\setkomafont{subsubsection}{\itshape\bfseries\normalsize} % 12. Italics para más contraste con el texto común.
\setkomafont{subsubsection}{\bfseries\normalsize} % 12. Les puse color, así que no hace falta que estén en cursiva.

\setkomafont{minisec}{\itshape\bfseries\normalsize}

%Máxima cantidad de niveles de títulos referenciados: 3 
%0. Capítulo / Título
%1. Sección
%2. Subsección
%3. Subsubsección
\setcounter{tocdepth}{2}
\setcounter{secnumdepth}{\subsectionnumdepth}

%Después de la tapa celeste con ventana provista por la secretaría del Departamento de  Electrónica  y  Automática,  sigue  una  hoja  impresa  provista  por  la Comisión  de Trabajos Finales donde, en el espacio correspondiente a la ventana debe escribirse el título de trabajo final, el nombre de los autores y el año. 
 
%A continuación sigue una hoja donde se escribe: Universidad Nacional de San Juan,Facultad  de  Ingeniería,  Departamento  de  Electrónica  y  Automática,  el  título  del trabajo final, el nombre de los autores, el nombre de los asesores y el año. Luego sigue una hoja con los agradecimientos. En otra hoja se inicia la escritura del índice. Finalmente sigue el resto del trabajo. 


% Otros paquetes útiles 


% The standard graphics inclusion package
\usepackage{graphicx}
% Set up how figure and table captions are displayed
%\usepackage{caption}
%\captionsetup{%
%  font=footnotesize,% set font size to footnotesize
%  labelfont=bf % bold label (e.g., Figure 3.2) font
%}
\graphicspath{{./imgs/}}


\addtokomafont{caption}{\small}
\addtokomafont{captionlabel}{\bfseries}


\usepackage{xcolor}
\definecolor{colortitulos}{RGB}{0,20,75}
\addtokomafont{sectioning}{\color{colortitulos}}

%csquotes
% uso: \enquote{text}
% ¿Por qué strict?
\usepackage[strict=true,autostyle=true]{csquotes}


% Add todo notes in the margin of the document
\usepackage[
%  disable, %turn off todonotes
  colorinlistoftodos, %enable a coloured square in the list of todos
  textwidth=\marginparwidth, %set the width of the todonotes
%  textsize=scriptsize, %size of the text in the todonotes
  textsize=tiny,
  obeyFinal, % 
  spanish,
  ]{todonotes}
  

% Make the standard latex tables look so much better
% An extended implementation of the array and tabular environments which extends the options for column formats, and provides "programmable" format specifications.
% The pack­age en­hances the qual­ity of ta­bles in LaTeX, pro­vid­ing ex­tra com­mands as well as be­hind-the-scenes op­ti­mi­sa­tion. Guide­lines are given as to what con­sti­tutes a good ta­ble in this con­text. 
\usepackage{array,booktabs}


\usepackage[allowlitunits]{siunitx}
%\selectlanguage{spanish} % También está como opción global de clase de documento..
\sisetup{group-digits = true}
\sisetup{group-minimum-digits = 5} % hasta 4 no separa de a 3.
\sisetup{group-separator = {\,}} % Un espacio pequeño
\sisetup{output-decimal-marker = {,}} % La coma para separar enteros de decimales
\DeclareSIUnit\pixel{px}







\usepackage{microtype}

\usepackage{bm} %bold symbols in math mode

\usepackage{multirow}

\usepackage{tabulary}
%\usepackage{tabularx}

\usepackage{pdflscape}
\usepackage{afterpage}

\KOMAoptions{toc=bibliography}

\usepackage{subcaption}

 \interfootnotelinepenalty=10000 %% Completely prevent breaking of footnotes
 
 
   \usepackage{url} % Ver si es prescindible; tal vez con hyperref basta-
 
 \usepackage[backend=biber,style=numeric,sorting=none,backref=true,hyperref=true]{biblatex}
\bibliography{tfinal.bib} %PUEDE HABER MÁS DE UNO?
 \renewcommand{\finentrypunct}{} % para que no ponga un punto al final de cada entrada de la bibliografía. Tal vez no importa si hay hipervínculos a las urls.
 \DefineBibliographyStrings{spanish}{%
   backrefpage = {Citado en página},% originally "cited on page" vid.
   backrefpages = {Citado en páginas},% originally "cited on pages"
 }
  
  

%  
%\usepackage[hidelinks]{hyperref}
%\hypersetup{%
% pdfpagelabels=true,%
% plainpages=false,%
% pdfauthor={Pablo Daniel AGUADO},%
% pdftitle={Una estrategia para la clasificación óptica de almendras},%
% pdfsubject={Trabajo Final, UNSJ},%
% bookmarksnumbered=true,%
% colorlinks=false,%
% citecolor=black,%
% filecolor=black,%
% linkcolor=black,% you should probably change this to black before printing
% urlcolor=black,%
% pdfstartview=FitH%
%}
  
\usepackage[hidelinks]{hyperref}
\hypersetup{%
  pdfpagelabels=true,%
  plainpages=false,%
  pdfauthor={Pablo Daniel AGUADO},%
  pdftitle={Una estrategia para la clasificación óptica de almendras},%
  pdfsubject={Trabajo Final, UNSJ},%
  bookmarksnumbered=true,%
  pdfstartview=FitH%
}%falta índice en marcadores.
 
 % Para otra vez: Usar \autoref{label} de hyperref, en lugar de \ref{}

% Separación en sílabas arbitraria
% see, e.g., https://en.wikibooks.org/wiki/LaTeX/Text_Formatting#Hyphenation
% for more information on word hyphenation

\hyphenation{MiPyME MiPyMEs}%

% Macros
% Macros de estilo
% Definimos nuevos comandos para facilitar la uniformidad de estilos.
% Ayudará, por supuesto, crear atajos de teclado que los introduzcan, o
% tener un editor que ayude con el autocompletado.

% Palabras en inglés: \ingles
% Definimos tanto el estilo general como el idioma. Lo del idioma es necesario
% porque el paquete de idioma define muchas cuestiones. Ver la documentación
% del paquete Polyglossia.
% Usamos cursivas (_italics_) para los extranjerismos.
\newcommand{\ingles}[1]{\textit{\texten{#1}}}


% Nombres propios: \nombre
% Ya sean personas o instituciones. Nombres no genéricos. Títulos de libros o documentos no.
\newcommand{\nombre}[1]{\textsf{#1}}

\newcommand{\programa}[1]{\texttt{#1}}

\def\td{\todo[inline]}



%\begin{figure}[hbtp]
%\centering
%\includegraphics[width=0.4\linewidth]{dummy}
%\caption[]{\td{Cita}}
%\label{fig:}
%\end{figure}


\def\imagenchica{5cm}
\def\imagenmedia{10cm}
\def\imagengrande{15cm}


\newcommand{\criterio}[3]{%
  \noindent
  \begin{minipage}{0.59\textwidth}
  \begin{description}
  \item[#1] #3
  \end{description}
  \end{minipage}\hfill
  \begin{minipage}{0.4\textwidth}
  #2
  \end{minipage}
  \medskip
  }
  
\newcommand{\criterioo}[3]{%
  \noindent
  \begin{minipage}{0.59\textwidth}
    \noindent
    \begin{description}
      \item[#1] #3
    \end{description}
  \end{minipage}\hfill
  \begin{minipage}{0.4\textwidth}
    \noindent
    \centering
    \includegraphics[]{imgs/#2}
  \end{minipage}
  \bigskip
}
  
\newcommand{\criteriooo}[4]{%
  \noindent
  \begin{minipage}{0.59\textwidth}
    \noindent
    \begin{description}
      \item[#1] #4
    \end{description}
  \end{minipage}\hfill
  \begin{minipage}{0.4\textwidth}
    \noindent
    \centering
    \includegraphics[]{imgs/#2}
    \includegraphics[]{imgs/#3}
  \end{minipage}
  \bigskip
}

\def\escaladiagramas{0.25}

%\includegraphics[width=\textwidth,height=0.9\textheight,keepaspectratio]{SegmentadorBorde1.png}

%  Configuración de página de título. No está implementado todavía.
%\title{Trabajo final muy interesante}
%\author{Nombre Nombre APELLIDO}



%  ---------------------------------------------------------------------------------------
%  DOCUMENTO
\begin{document}


%    PRELIMINARES / FRONT MATTER (tapa, título, resumen, agradecimientos, índices, etc)

% Usamos números romanos en los preliminares. Cada vez que lo cambiamos, empieza desde el
% inicio. Usamos el comando \pagenumbering{numstyle}
\pagenumbering{roman}


%      TAPA
% No está implementada. Usar 'Tapa con logo Trab.Final.doc'


%      PÁGINA DE TÍTULO
% Comando para crear la página de título automáticamente, según la clase de documento y
% plantilla que se usen. No lo usamos todavía; en su lugar, incluimos una página que 
% se debe editar manualmente.
%\maketitle


% Página de título que se debe editar manualmente. 
\begin{titlepage}
	{\centering
	\includegraphics[width=0.2\textwidth]{unsj}
	
	\vspace{1cm}
	
	{\scshape\LARGE Universidad Nacional de San Juan}
	
%	{\scshape\large Facultad de Ingeniería - Ingeniería Electrónica}
	{\scshape\LARGE Facultad de Ingeniería}
	
	\vspace{1cm}
	
	{\scshape\Large Trabajo Final de Graduación}
	
	\vspace{1.5cm}
	
%	{\LARGE\bfseries Una estrategia para la clasificación\\óptica de almendras}
	{\Huge\bfseries Plantilla de \LaTeX para los trabajos\\finales de Ingeniería Electrónica}
		
	\vspace{2cm}
	
%	{\Large\itshape Pablo Daniel AGUADO}
	{\Large Edwin BARRAGÁN\par}
	{\Large Pablo Daniel AGUADO\par}
	\vfill}

%	supervised by\par
%	Dr.~Mark \textsc{Brown}

\centering
		{\textsc{Asesores:}}
		
{Dr.~Ing.~Germán Alejandro GONZÁLEZ}
		
{Dr.~Ing.~Germán Emilio MAS}

{Dr.~Ing.~Fabricio \enquote{Carlos} EMDER}

	
	\vfill
	
	% Bottom of the page
	\centering
	{\large2017}
\end{titlepage}


%      COSAS PENDIENTES
% Una lista de cosas para hacer. Depende del paquete 'todonotes'. Se dejará de imprimir
% si se usa la opcion 'final' en la definición de clase del documento:
%   \documentclass[final,spanish,twoside=no]{scrreprt}
% Ver en 'preambulo_UNSJ.tex'
\listoftodos  % Comando para incluir la lista de cosas para hacer.


%      RESUMEN / ABSTRACT
\begin{abstract}
\todo{Provisorio}

En este trabajo se diseñó y creó un programa informático para la clasificación automática de almendras peladas a partir de imágenes de ellas, analizando diversas características de forma y de color. Para probarlo se construyó un prototipo de sistema de visión artificial con el cual se creó un conjunto de \num{564} imágenes de almendras y otros objetos. El conjunto de imágenes fue etiquetado manualmente en base a las normas de la Comisión Económica de las Naciones Unidas para Europa (UNECE) para el comercio de almendras. Los resultados del clasificador desarrollado son similares a los obtenidos con algoritmos de clasificación estándar, como máquinas de soporte vectorial o \ingles{Boosted Trees}. Los descriptores elegidos permiten clasificar binariamente el conjunto con una exactitud global de \SI{93}{\percent}.

\end{abstract}



%      AGRADECIMIENTOS
\chapter*{Agradecimientos}
  Acá le agradezco a todos los miembros de la prestigiosa y gloriosa Comisión de Trabajo Final por sus incontables aportes a la causa.
  Si pongo punto y meto enter no se vé en el documento.\\
  Si escribo barra barra hago un salto de linea pero no cambio de párrafo.

  Si doy doble enter, coloca sangría, pero no hace el salto de línea para el párrafo.\\

  Este último sí que es un párrafo decente!



% Los tipos de página se pueden cambiar con \pagestyle{option} y con
% \thispagestyle{option} para una sola página.
% Los tipos de página para esta plantilla con con Koma-Script son:
%   empty                   vacía
% plain.scrheadings  sólo números de página
%   scrheadings         números de página y cabecera
%
% Por defecto, el cuerpo usa plain.scrheadings, por lo que tenemos que decir que 
% queremos tener cabeceras también, incluso en los índices.
\pagestyle{scrheadings} 


%      ÍNDICES
% Índice normal / tabla de contenidos
\tableofcontents


% Índice de tablas (innecesario para nuestro trabajo final)
%\listoftables

% Índice de figuras (innecesario para nuestro trabajo final)
%\listoffiguress


%      PÁGINA DE DEDICATORIAS O SIMILAR (opcional)
\clearpage
\thispagestyle{empty}
%
\null\vfill
%
{
\begin{center}
\noindent
\centering
%
\parbox{0.8\textwidth}
{
\itshape
%\centering
{
\raggedright
En El Principito de Antoine de Saint-Exupéry, el protagonista encuentra a un hombre de negocios que acumula estrellas con el único propósito de poder comprar más estrellas.

\bigskip

El Principito está perplejo. Solo posee una flor, que riega a diario, y tres volcanes que limpia cada semana.

\bigskip

\enquote{Es útil, pues, para mis volcanes y para mi flor que yo las posea}, dijo el Principito, \enquote{pero tú, tú no eres nada útil para las estrellas\textellipsis}.
}

\bigskip

\raggedleft{\url{http://custodians.online/spanish3.html}}\par%
}
\end{center}
}
\vfill

% Con el comando \clearpage aseguramos de terminar la página e imprimir todos
% los elementos flotantes (floats) que no se habían impreso aún.
\clearpage 



%    CUERPO PRINCIPAL
% Acá incluímos todos los capítulos, menos los Anexos, si los hubiese.

% Usamos números arábigos en el cuerpo principal.
\pagenumbering{arabic}

%      CAPÍTULOS
% Capítulos de ejemplo. No hay ninguna estructura establecida.
\chapter{Lecciones de la vida}

\section{Referidas a los alimentos}

\subsection{Burbujas quemadas de las galletas de agua}

\subsubsection{Por qué son malas}

\subsubsection{Cómo sacarlas}

\subsubsection{Análisis de mercado}

\minisec{Traviata} % Minisec es de las clases de Koma-Script

\minisec{Criollitas}

\minisec{Mediatarde}

%\input{secciones/conclusiones.tex}



%    BIBLIOGRAFÍA / REFERENCIAS
% Las referencias están en un archivo '.bib' con formato BibLaTeX.
% Declaramos los archivos en el preámbulo (fuera del entorno 'document'):
% ver en 'preambulo_otros.tex'

% Normalmente, los comandos de imprimir bibliografía sólo incluiran aquellas referencias
% que fueron citadas en el texto. Si queremos incluir todas, aunque no estén citadas, 
% usamos el comando \nocite{*}.
%Incluimos en la bibliografía todas (*) las referencias no citadas en el texto.
%\nocite{*} 

% REVISAR QUE SE IMPRIMAN EN EL ÍNDICE
% Este comando imprime toda la bibliografía. Ver más en la documentación de BibLaTeX.
%\printbibliography
%\printbibliography[title={Referencias}]

% Otra opción es ordenarlas arbitrariamente. En este caso, decidí usar tres categorías:
% referencias generales, programas y créditos de imágenes.
% Para ello, cada elemento del archivo '.bib' debe tener una clave 'keyword'.
\printbibliography[keyword={referencias},title={Referencias}]
\printbibliography[keyword={software},title={Programas}]
\printbibliography[keyword={imagenes},title={Créditos de imágenes}]



%      APÉNDICES / ANEXOS
% Usamos el comando \appendix, que declara que los capítulos siguientse son apéndices
% y por tanto los enumera de forma distinta.
\appendix

% Incluimos los archivos de anexos
\chapter{Organización de los archivos}\label{anexo1}

\begin{labeling}[—]{\texttt{/tfinal\_AGUADO.pdf}}
\setkomafont{labelinglabel}{\ttfamily}
\item [/inicializar.m] Archivo de inicialización (ver anexo \ref{anexo2}).
\item [/tfinal\_AGUADO.pdf] Este documento.
\item [/clases/] Archivos de clase de preprocesadores, segmentadores, clasificadores, clases principales, enumeraciones y otros.
\item [/experimentos/] Archivos de cada experimento realizado. Cada carpeta contiene los archivos de configuración del experimento y planillas con los resultados generados.
\item [/funciones/] Funciones principales y auxiliares desarrolladas en \nombre{Matlab}.
\item [/GUI/] Interfaz gráfica de usuario.
\item [/imagenes/] Los conjuntos de imágenes generados y sus metadatos. \nombre{set1} sirvió para experimentación inicial, con \nombre{set2} se desarrollaron los primeros algoritmos y \nombre{set3} es el conjunto principal con el que se evaluó el sistema (ver sección \ref{captura:conjuntodeimagenes}). De este último se incluyen versiones preprocesadas ---recortadas, sin distorsión y con máscaras binarias---. Los archivos \texttt{.csv} son las listas completas de imágenes y de los conjuntos de entrenamiento y evaluación; la planilla \texttt{.xlsx} tiene información sobre la clasificación manual.\todo{NO OLVIDAR ADJUNTAR}
\item [/info/] Documentos e imágenes que fueron útiles para la realización de este trabajo.\todo{NO OLVIDAR ADJUNTAR}
\item [/otros/] Archivos correspondientes a programas de terceros requeridos o utilizados: \nombre{Balu}, \nombre{YAMLMatlab} y \nombre{multic}.
\item [/scripts/] Programas auxiliares usados durante el trabajo. No son necesarios para la ejecución de experimentos; sirvan para referencia futura.
\end{labeling}
\chapter{Uso de los programas}\label{anexo2}

Los programas requieren que se definan las variables \texttt{carpeta\_base} y \texttt{carpeta\_imagenes} con las rutas \textbf{absolutas} a la carpeta raíz de los programas (raíz de las carpetas \texttt{GUI}, \texttt{clases}, \textellipsis) y a la carpeta raíz de las imágenes respectivamente (raíz de las carpetas \texttt{set1}, \texttt{set2}, \texttt{set3}, \textellipsis).


Para ello, como requisito inicial, se debe editar el archivo \texttt{inicializar.m}, disponible en el directorio raíz.

Si bien se pueden editar y ejecutar los experimentos disponibles en la carpeta \texttt{/experimentos}, recomiendo ejecutar la interfaz gráfica de usuario para poder apreciar lo que están haciendo los programas. Para ello:

\begin{enumerate}
\item Editar y ejecutar \texttt{inicializar.m}.
\item Editar \texttt{/GUI/gui.m} y seleccionar (comentando y descomentando líneas) el conjunto de imágenes a utilizar y las etapas del algoritmo.
\item Ejecutar \texttt{gui.m}.
\end{enumerate}




\end{document}